\documentclass{article}

%\input{commands}

% Page Layout
\setlength{\oddsidemargin}{0.0in}
\setlength{\textwidth}{6.5in}
\setlength{\textheight}{8in}
%\parindent 0in
%\parskip 12pt

% Set Fancy Style
%\pagestyle{fancy}

%\lhead{Left Header}
%\rhead{Right Header}

\begin{document}

% Change Font and Spacing
\large % change the font size to 12pt
% \linespread{1.1} % change the line spacing

% Set the counter
\setcounter{section}{0}

\section{Benchmarking}

\begin{table}
\centering
\begin{tabular}{c c c c c c c c c}
Method & Data set & $n$ & $p$ & Time (s) & Ave. ESS & SD ESS & Ave. ESR & SD ESR \\
Tri. & BHI & 506 & 103 & 15.39 & 258  &  291 &    17 &  19 \\
Stb. & --- & --- & --- & 19.56 & {\bf 5730} & 2681 & {\bf 293} & 137 \\

Tri. & BH  & 506 & 13  &  0.22 & 4036 & 1888 & {\bf 18278} & 8550 \\ 
Stb. & --- & --- & --- &  0.61 & {\bf 6615} & 2488 & 10887 & 4094 \\

Tri. & DBT & 442 & 10  &  0.15 & 1978 & 1089 & {\bf 12978} & 7144 \\
Stb. & --- & --- & --- &  0.49 & {\bf 5989} & 2306 & 12152 & 4679 \\

\multicolumn{9}{c}{Orthogonalized Design Matrix} \\

Tri. & BHI & 506 & 103 &  1.62 & 6435 & 1814 &  {\bf 3975} & 1120 \\
Stb. & --- & --- & --- &  4.17 & {\bf 7690} & 1676 &  1844 &  402 \\

Tri. & BH  & 506 & 13  &  0.16 & 7972 &  930 & {\bf 49561} & 5782 \\ 
Stb. & --- & --- & --- &  0.56 & {\bf 9429} &  794 & 16821 & 1416 \\

Tri. & DBT & 442 & 10  &  0.12 & 7026 & 1203 & {\bf 57915} & 9918 \\
Stb. & --- & --- & --- &  0.41 & {\bf 8644} & 1059 & 20876 & 2558 \\

\end{tabular}
\end{table}

Tri. refers to the mixture of triangles method and Stb. refers to the precision
mixture of normals method.  BHI refers to the Boston Housing data with
interactions and squared terms, BH refers to the Boston Housing data without any
additional terms, and DBT refers the the diabetes data found in Efron's LARS
package. For each data set and each method, 10,000 samples of $\beta$ were
generated after an initial burn-in of 2,000.  The parameter $\sigma^2$ was given
a Jeffrey's prior while the prior for $\tau = \nu^{-1/\alpha}$ was induced by
placing a $\textmd{Ga}(2, \textmd{rate}=2)$ prior on $\nu$.  We calculated the
effective sample size for each component of $(\beta_i)_{i=1}^p$ and summarized
the results via the mean and standard deviation.  This process was repeated 10
times; the mean quantities are recorded above.  We report the same results for
the effective sampling rate, which is the effective sample size per second of
run time.  Time is the execution time of our code while $n$ is the number of
observations and $p$ is dimension of $\beta$.  The code was written using a C++
wrapper function called through R using a MacBook Pro with 8GB of RAM and a 2
GHz Intel Core i7 processor.  The exponentially stilted stable distributions
were sampled using Luc Devroye's algorithm using code from the \texttt{copula}
package.

In addition to the general regression setting, we benchmarked each method when
the design matrix is orthogonal.  In that case, one may sample from the
truncated multivariate Gaussian distribution $(\beta | y, \ldots)$ very quickly
when using the mixture of triangles method.  For a fair comparison we also
implemented a special routing to sample from $(\beta | y, \ldots)$ that takes
advantage of the structure of the problem when using the mixture of normals
method.  The same priors were used for $\sigma^2$ and $\tau$.

The stable method always has a better effective sample size; however, the
mixture of triangles consistently has better effective sampling rates, which is
the speed at which effective samples are generated.  In the orthogonal setting,
the mixture of triangles method is more competitive in terms of effective sample
size and superior in terms of effective sampling rate. We take this to be
evidence that the mixture of triangles may in some situations generate effective
samples more quickly than the precision mixture of normals.  However, there are
two caveats to this conclusion.  First, this is an aggregate measure and does
not consider the variation in the effective sample sizes of the components of
$\beta$.  Second, the effective sampling rate is a rather imprecise quantity
since there are many factors that could affect the speed at which an algorithm
runs.  The language, implementation, compiler options (when working in C, C++,
or Fortran), and hardware all play a part in determining the runtime.  The above
analysis does not examine these components of variation in the effective
sampling rate; however, we did do our best to ensure that both the mixture of
triangles method and the mixture of normals method were compared on equal
footing.  Ultimately, the effective sampling rates reported above are specific
to the \texttt{BayesBridge} R package that we used to benchmark both methods.

% If you have a bibliography.
% The file withe bibliography is name.bib.
% \bibliography{name}{}
% \bibliographystyle{plain}

\end{document}
